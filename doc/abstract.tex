%%%%%%%%%%%%%%%%%%%%%%%%%%%%%%%%%%%%%%%%%%%%%%%%%%%%%%%%%%%%%%%%%
% Dissertacao de Mestrado / Dept Fisica, CFM, UFSC              %
% Andre@UFSC - 2011                                             %
%%%%%%%%%%%%%%%%%%%%%%%%%%%%%%%%%%%%%%%%%%%%%%%%%%%%%%%%%%%%%%%%%

%***************************************************************%
%                                                               %
%                          Abstract                             %
%                                                               %
%***************************************************************%

\begin{abstract}[Abstract]
  \phantomsection
  \addcontentsline{toc}{chapter}{Abstract}


Astronomical mega-surveys are creating a large volume of data, creating new
challenges regarding data management and archiving. Data mining techniques are
needed to make use of this data deluge. The use of Relational Database
Management Systems (RDBMS) for management and data access in large surveys is
explored in this study. The \starlight catalog is imported into a RDBMS, using
{\em CasJobs} for data access. These data are produced feeding about
$1\,000\,000$ galaxy spectra from Sloan Digital Sky Survey (\SDSS) and a set of
simple stellar populations to \starlight, yielding a catalog of physical
properties of those galaxies. This analysis is based entirely on data from the
optical spectrum, serving as a benchmark for studies in other wavelengths. The
Galaxy Evolution Explorer (\galex) latest data release (GR6) includes a
crossmatch between GALEX and SDSS catalogs. Data from \starlight are then
matched to GALEX, adding ultraviolet photometry to the \starlight catalog. As a
case study, a redshift-limited sample is taken from this new catalog, galaxies
are classified based on the $\nII/\Halpha$ (WHAN) diagram, and the physical
properties for each class are analyzed against their ultraviolet colours.

\end{abstract}

% End of abstract



