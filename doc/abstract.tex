%%%%%%%%%%%%%%%%%%%%%%%%%%%%%%%%%%%%%%%%%%%%%%%%%%%%%%%%%%%%%%%%%
% Dissertacao de Mestrado / Dept Fisica, CFM, UFSC              %
% Andre@UFSC - 2011                                             %
%%%%%%%%%%%%%%%%%%%%%%%%%%%%%%%%%%%%%%%%%%%%%%%%%%%%%%%%%%%%%%%%%

%***************************************************************%
%                                                               %
%                          Abstract                             %
%                                                               %
%***************************************************************%

\begin{abstract}

Astronomical megasurveys are creating a large volume of data. New techniques of
data management and data mining are needed to make use this deluge of data. The
use of Relational Dabase Management Systems (RDBMS) for management and data
access for large surveys is explored. The \starlight catalog is imported to a
RDBMS, using {\em CasJobs} for data access. \starlight takes galaxy spectra from
Sloan Digital Sky Survey (SDSS) and a set of simple stellar populations,
yielding a catalog of physical properties from those galaxies. This analisis is
based entirely in the optical spectrum, serving as a benchmark when studying
those galaxies in new wavelengths. The Galaxy Evolution Explorer (GALEX) latest
data release (GR6) includes a crossmatch between GALEX and SDSS objects. Data
from \starlight is then matched to GALEX, adding ultraviolet photometry to the
\starlight catalog. As a case study, a redshift-limited sample is taken from
this new catalog, galaxies are classified based on the WHAN diagram, and the
phyisical properties for each class is analyzed against theis ultraviolet
colors.

\end{abstract}

% End of abstract



