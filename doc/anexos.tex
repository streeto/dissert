%%%%%%%%%%%%%%%%%%%%%%%%%%%%%%%%%%%%%%%%%%%%%%%%%%%%%%%%%%%%%%%%%
% Dissertacao de Mestrado / Dept Fisica, CFM, UFSC              %
% Andre@UFSC - 2011                                             %
%%%%%%%%%%%%%%%%%%%%%%%%%%%%%%%%%%%%%%%%%%%%%%%%%%%%%%%%%%%%%%%%%


%:::::::::::::::::::::::::::::::::::::::::::::::::::::::::::::::%
%                                                               %
%                            Anexos                             %
%                                                               %
%:::::::::::::::::::::::::::::::::::::::::::::::::::::::::::::::%

%***************************************************************%
%                                                               %
%                    Manual Starlight + Galex                   %
%                                                               %
%***************************************************************%

\chapter{{\em Queries} SQL}
\label{apendice:Queries}

{\em Queries} SQL utilizadas no texto. 

\begin{figure}
	\begin{Verbatim}[commandchars=\\\{\}]
	\textbf{UPDATE} sample
		\textbf{SET} SpecObjID=so.SpecObjID, ObjID=so.BestObjID
	\textbf{FROM} sample s2 \textbf{INNER JOIN} DR7..SpecObjAll so
		\textbf{ON} so.MJD=s2.MJD
		\textbf{AND} so.Plate=s2.Plate
		\textbf{AND} so.FiberID=s2.FiberID
	\end{Verbatim}
	\caption
	[{\em Query} para atualizar os índices da amostra de galáxias do
	\starlight.]
	{Atualização dos índices da amostra de galáxias do \starlight. A {\em query}
	foi executada no {\em CasJobs} do \SDSS DR7 para obter {\tt SpecObjID} e {\tt
	BestObjID} dado o tripleto [{\tt MJD}, {\tt Plate}, {\tt FiberID}].}
	\label{fig:AtualizaObjIds}
\end{figure}


\begin{figure}
	\begin{Verbatim}[commandchars=\\\{\}]
	\textbf{SELECT INTO} mydb..galex_ais
		s.objid \textbf{AS} sdssobjid, x.objid \textbf{AS} galexobjid,
		s.mjd, s.plate, s.fiberid,
		g.fuv_mag, fuv_magErr,
		g.nuv_mag, g.nuv_magErr,
		g.e_bv,
		g.band,
		x.distance,
		pe.fexptime,
		pe.nexptime
	\textbf{FROM} mydb..sample s
	\textbf{LEFT JOIN} xSDSSDR7 x
		\textbf{ON} s.objid = x.sdssobjid
		\textbf{AND} x.distanceRank=1
		\textbf{AND} x.reverseDistanceRank=1
		\textbf{AND} x.multipleMatchCount=1
		\textbf{AND} x.reverseMultipleMatchCount=1
	\textbf{LEFT JOIN} photoobjall g
		\textbf{ON} g.objid = x.objid
	\textbf{LEFT JOIN} photoextract e
		\textbf{ON} e.photoextractid=g.photoextractid
	\textbf{WHERE} e.mpstype='AIS'
	\end{Verbatim}
	\caption[{\em Match} entre os objetos da amostra do \starlight e \galex.]
	{{\em Query} para o {\em match} entre os objetos da amostra do \starlight e
	\galex AIS. A mesma {\em query} foi usada para o MIS, trocando apenas o nome da
	tabela para {\tt galex\_mis} e modificando a última linha para {\tt
	e.mpstype='MIS'}.}
	\label{fig:QueryMatchAIS}
\end{figure}

% TODO: Caption da figura fig:QuerySampleAIS
\begin{figure}
	\begin{Verbatim}[commandchars=\\\{\}]
	\textbf{SELECT INTO} MyDB..galex_ais_elines_z
		s.specobjid,
		g.fuv_mag \textbf{AS} FUV, g.nuv_mag \textbf{AS} NUV,
		o.Mu \textbf{AS} u, o.Mg \textbf{AS} g, o.Mr \textbf{AS} r,
		o.Mi \textbf{AS} i, o.Mz \textbf{AS} z,
		s.mcor_gal, s.at_flux, s.at_mass, s.am_flux,
		s.am_mass, s.AV,
		e.oiii_5007_flux, e.oiii_5007_flux_err,
		e.oiii_5007_ew, e.oiii_5007_ew_err, e.oiii_5007_sn,
		e.nii_6584_flux, e.nii_6584_flux_err,
		e.nii_6584_ew, e.nii_6584_ew_err, e.nii_6584_sn,
		e.halpha_flux, e.halpha_flux_err,
		e.halpha_ew, e.halpha_ew_err, e.halpha_sn,
		e.hbeta_flux, e.hbeta_flux_err,
		e.hbeta_ew, e.hbeta_ew_err, e.hbeta_sn,
		o.z \textbf{AS} redshift
	\textbf{FROM} galex_ais g
	\textbf{INNER JOIN} synthesis_results s\textbf{ON}
		s.specobjid = g.specobjid
	\textbf{INNER JOIN} el_fit_all e \textbf{ON}
		s.synid = e.synid
	\textbf{INNER JOIN} obs_parameters o \textbf{ON}
		o.specobjid = s.specobjid
	\textbf{WHERE} g.galexobjid <> 0
	\end{Verbatim}
	\caption[Extração da amostra de galáxias.]
	{TODO: Extração da amostra de galáxias.}
	\label{fig:QuerySampleAIS}
\end{figure}


% End of this chapter
