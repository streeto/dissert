%%%%%%%%%%%%%%%%%%%%%%%%%%%%%%%%%%%%%%%%%%%%%%%%%%%%%%%%%%%%%%%%%
% Dissertacao de Mestrado / Dept Fisica, CFM, UFSC              %
% Andre@UFSC - 2011                                             %
%%%%%%%%%%%%%%%%%%%%%%%%%%%%%%%%%%%%%%%%%%%%%%%%%%%%%%%%%%%%%%%%%

%:::::::::::::::::::::::::::::::::::::::::::::::::::::::::::::::%
%                                                               %
%                          Capítulo 1                           %
%                                                               %
%:::::::::::::::::::::::::::::::::::::::::::::::::::::::::::::::%

%***************************************************************%
%                                                               %
%                         Introdução                            %
%                                                               %
%***************************************************************%

\chapter{Introdução}
\label{sec:Intro}


\section{A nova era da astronomia}

Com o advento dos {\em mega-surveys}, está começando uma revolução na forma de
fazer ciência na astronomia. Os diversos {\em surveys}\footnote{Um {\em Survey}
astronômico é um levantamento ou mapeamento de regiões do céu utlizando
telescópios.} em execução atualmente estão produzindo dados a uma taxa da ordem
de petabytes por ano. Este talvez seja o primeiro campo da ciência onde as
informações coletadas por máquinas tenham -- nas próximas décadas -- um volume
maior do que os seres humanos são capazes de digerir. Uma espécie de {\em
Singularidade Tecnológica}\footnote{O termo {\em Singularidade tecnológica} se
refere a um futuro hipotético onde uma inteligência superior à humana emerge
através da tecnologia. Qualquer previsão após tal fato se torna muito difícil,
algo similar a um horizonte de eventos, dada a dificuldade em entender uma
inteligência superior à humana.} da astrofísica, onde máquinas coletam, analizam
e classificam dados. O papel do cientista num cenário como este ainda não está
muito claro \citep{Norris2010}.

O {\em Sloan Digital Sky Survey}, (\SDSS) \citep{York2000} é referência quando
falamos em {\em surveys} modernos. Em seus 11\fixme anos de funcionamento,
obteve imagens em 5 filtros de um quarto do céu e espectros de um milhão de
galáxias. O seu catálogo contém 4 terabytes de fotometria e espectros, sem
contar as imagens. O \SDSS foi praticamente o primeiro {\em survey} a conseguir
popularizar o acesso aos seus dados. Os seus dados foram feitos públicos desde o
início\citneed, gerando uma ``corrida do ouro'' no seu vasto volume de dados.
Esta filosofia é compartilhada hoje pela grande maioria dos {\em surveys} de
grande porte.

Existem diversos {\em surveys} em operação atualmente. O {\em Wide-field
Infrared Survey Explorer} (WISE) é um telescópio espacial da NASA\footnote{{\em
NASA Explorer Mission} - \url{http://explorers.gsfc.nasa.gov/missions.html}},
que está mapeando o céu inteiro nas faixas de $3,4$, $4,6$, $12$ e $22\mu$m do
infravermelho \citep{Wright2010}. O {\em Visible and Infrared Survey Telescope
for Astronomy} (VISTA) é um telescópio no Chile fazendo um {\em survey} do céu
do hemisfério sul no infravermelho próximo \citep{Born2010}. Tratado com mais
detalhes no capítulo \ref{sec:Galex}, o {Galaxy Evolution Explorer} (\galex)
mapeou o céu em ultravioleta. O Kepler é um telescópio espacial da NASA
\citep{Borucki2010}, está fazendo um survey de uma região da Via Láctea para
descobrir a fração de estrelas com planetas similares à Terra na nossa galáxia.
Convém mencionar também o {\em 2dF Galaxy Redshift Survey} (2dFGRS)
\citep{Colless1999} e o {\em Two Micron All Sky Survey} (2MASS)
\citep{Skrutskie2006}. Embora estes {\em surveys} já tenham sido concluídos, os
seus dados permanecem disponíveis publicamente.

% TODO: Parágrafo sobre J-PAS.
TODO: Parágrafo sobre J-PAS.

O LSST mapeará metade do céu aproximadamente a cada mês, durante cerca de dez
anos. Serão mais de um petabyte em imagens brutas por ano, muito mais do que
poderia ser revisado por humanos.

% TODO: Observatórios virtuais.
TODO: Observatórios virtuais. Datamining.

FIXME: Os cientistas não armazenam todos os dados das observações, mas através
de serviços de acesso manipulam estes dados, podendo também fazer a combinação
destes com outros bancos de dados complementares, e retirando as informações
relevantes à pesquisa realizada.

(\ldots)

Há pouco mais de uma década astrônomos extragaláticos conheciam os seus objetos
de estudo pelo nome. Gráficos contendo algumas dúzias de galáxias eram objeto de
luxo. Tradicionalmente um astrônomo faz um pedido de tempo de um dado
observatório, obtém os dados (em geral imagens e espectros) e os leva para casa
para seu uso individual\footnote{Alguns observatórios mantém arquivos das
observações\citneed, mas a organização dos dados em geral fica muito aquém do
desejável.}. Isto não é uma crítica a esta abordagem, muito pelo contrário.
Alguns programas observacionais certamente são efetuados mais adequadamente
desta forma. O que se está advocando é a coleta sistemática de dados, tal que
esses dados possam ser posteriormente utilizados para diversos fins. Ao invés de
ser um substituto, um {\em mega-survey} é um complemento à observação
tradicional.


%***************************************************************%
%                                                               %
%                 Introdução - STARLIGHT + SDSS                 %
%                                                               %
%***************************************************************%

\section{\starlight e o \SDSS}
\label{sec:Intro:Starlight}

% TODO: Intro = falar do sucesso do \starlight. Estudo das propriedades das
% galaxias no UV.

O \starlight refina dados do \SDSS e gera um catálogo de propriedades físicas
das galáxias.

Noção geral de como o \starlight funciona. Síntese de população estelar.

Resultados do \starlight.


\begin{verbatim}

Um exemplo bastante
conhecido é programa STARLIGHT, que tem como entrada uma base de populações
estelares simples e espectros de galáxias do Sloan Digital Sky Survey (SDSS), e
gera como saída informações relacionadas ao histórico de formação estelar das
galáxias. Baseada inteiramente no espectro óptico, esta análise serve de guia
para interpretar dados em outras faixas espectrais para galáxias de diferentes
tipos (como as que estão formando estrelas, as galáxias ativas de diferentes
classes, as passivas e as aposentadas).

\end{verbatim}


\begin{verbatim}

Nos últimos anos, novas bibliotecas estelares com alta resolução espectral (~ 1
Å) foram incorporadas a modelos de síntese evolutiva, fornecendo o espectro
detalhado L(lambda,t,Z) de populacoes estelares simples em funcao de sua idade
(t) e metalicidade (Z). O artigo de Bruzual & Charlot (2003), um “best-seller”
desde sua publicação, é o exemplo mais conhecido desta nova geração de modelos,
mas outros conjuntos de bibliotecas estão surgindo (Le Borgne et al 2004;
González Delgado et al 2004; Sanchez-Blasquez et al 2006), e muitos outros estão
por vir (Charlot & Bruzual 2007; Vazdekis et al 2007; Coelho et al 2007; etc).
Ao contrário das técnicas tradicionais de caracterização de populações
estelares, que se baseiam em uns poucos índices espectrais escolhidos “a dedo”
(eg, Gallazi et al 2005), essas novas ferramentas permitem modelar todo espectro
observado Lobs(lambda), incorporando assim toda informacap disponivel, o que
naturalmente produz resultados mais robustos e detalhados. Na prática, o uso
efetivo desses novos modelos no estudo da história de formação estelar em
galáxias requer a implementação de técnicas matemáticas que permitam deduzir a
mistura de populações estelares formadas em diferentes tempos e suas respectivas
metalicidades a partir de um espectro observado Lobs(lambda). Este eh um
problema ao qual o Grupo de Astrofísica da UFSC se dedica já há alguns anos (eg,
Cid Fernandes et al 2001b; Leão 2001; Gomes 2005). A experiência acumulada nos
permitiu desenvolver um código deste tipo logo após a publicação dos modelos de
Bruzual & Charlot (2003). Os resultados são assustadoramente bons! A figura 1
abaixo ilustra a aplicação deste novo código de “síntese espectral” ao espectro
de duas galáxias observadas pela SDSS. (573140 outros exemplos estão disponíveis
em www.starlight.ufsc.br!)

Matematicamente, o problema de síntese, abordado sob uma ótica Bayesiana,
envolve o mapeamento de um vasto espaço de parâmetros, de dimensão da ordem de
50 – 100 (dependendo do grau de detalhamento no vetor de populações).  Nosso
código, batizado de STARLIGHT, varre e mapeia este espaço de parâmetros usando
uma combinação de simulated annealing com o algoritmo de Metropolis e métodos de
“Markov-Chain-Monte-Carlo” (MCMC), mais difundidos na comunidade de Física
Estatística, mas que estão ganhando terreno em aplicações astrofísicas diversas.
O STARLIGHT, fornece várias informações como saída: (a) a história de formação
estelar (que nos revela como se formou e evoluiu a galáxia), expressa na forma
de um “vetor de populações” x = x(t,Z), que fornece a fração da luz, ou massa,
proveniente de uma dada população (t,Z); a extinção (AV,devida a grãos de poeira
inter-estelar que provocam um enfraquecimento e avermelhamento da luz observada)
e a dispersão de velocidades (s*), que mede a “temperatura cinética” das
estrelas sob ação do potencial gravitacional da galáxia. Esses parâmetros
básicos podem ser combinados para produzir grandezas fisicamente interessantes,
como a massa estelar da galáxia (M*), sua idade média (t*) e metalicidade
estelar média (Z*). Além disso, o ajuste do espectro é tão bom que permite
medidas bem mais precisas das linhas de emissão a partir do espectro residual
L(lambda) = Lobservado(lambda) – Lmodelo(lambda), incluindo a detecção de linhas
fracas dificilmente mensuráveis no espectro total.

\end{verbatim}


%***************************************************************%
%                                                               %
%                     Introdução - GALEX                        %
%                                                               %
%***************************************************************%

\section{Buscando o ultravioleta}
\label{sec:Intro:UV}

% TODO: Intro - outras bandas espectrais.

A tendência é cobrir todo o espectro eletromagnético.

Qual ciência pode ser feita com cada faixa.

Por que UV? Limitações do UV.

Resumo do Galex, o que é, como funciona, motivação.


%***************************************************************%
%                                                               %
%                 Introdução - Este trabalho                    %
%                                                               %
%***************************************************************%

\section{Este trabalho}
\label{sec:Intro:EsteTrab}

% TODO: Intro - este trabalho.
TODO: Intro - este trabalho.

Crossmatch entre fontes SDSS do \starlight{} e do Galex.

Cores UV para os tipos de galáxias classificados pelo WHaN.

Propriedades físicas dos tipos de galáxias no diagrama cor-cor UV.

Organização deste trabalho.

% End of this chapter
