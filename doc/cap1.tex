%%%%%%%%%%%%%%%%%%%%%%%%%%%%%%%%%%%%%%%%%%%%%%%%%%%%%%%%%%%%%%%%%
% Dissertacao de Mestrado / Dept Fisica, CFM, UFSC              %
% Andre@UFSC - 2011                                             %
%%%%%%%%%%%%%%%%%%%%%%%%%%%%%%%%%%%%%%%%%%%%%%%%%%%%%%%%%%%%%%%%%

%:::::::::::::::::::::::::::::::::::::::::::::::::::::::::::::::%
%                                                               %
%                          Capítulo 1                           %
%                                                               %
%:::::::::::::::::::::::::::::::::::::::::::::::::::::::::::::::%

%***************************************************************%
%                                                               %
%                         Introdução                            %
% XXX: ~1 pagina por section.                                   %
%                                                               %
%***************************************************************%

\chapter{Introdução}
\label{sec:Intro}


\section{A nova era da astronomia}
% TODO: Nova era de bancos de dados em astronomia. Grandes bancos de dados.
% Datamining. Olhar o projeto do mestrado.

% TODO: Falar do papel do SLOAN. WISE, 2dF, GALEX.

%***************************************************************%
%                                                               %
%                 Introdução - STARLIGHT + SDSS                 %
%                                                               %
%***************************************************************%

\section{\starlight + \SDSS}
\label{sec:Intro:Starlight}

%TODO: Falar do sucesso do \starlight. Estudo das propriedades das galaxias
% no UV.


%***************************************************************%
%                                                               %
%                     Introdução - GALEX                        %
%                                                               %
%***************************************************************%

\section{GALEX}
\label{sec:Intro:Galex}

Resumo do Galex, o que é, como funciona, motivação.
Por que UV? Ir pra outros comprimentos de onda. Limitações do UV. Propaganda do
GALEX. Necessidade de ir para outros $\lambda$, e qual ciência pode ser feita
com cada faixa.



%***************************************************************%
%                                                               %
%           Introdução - STARLIGHT: Trab. anteriores            %
%                                                               %
%***************************************************************%

\section{Trabalhos Anteriores}
\label{sec:Intro:TrabAnt}

% XXX: later?
Observatórios virtuais. Crossmatch.


%***************************************************************%
%                                                               %
%                 Introdução - Este trabalho                    %
%                                                               %
%***************************************************************%

\section{Este Trabalho}
\label{sec:Intro:EsteTrab}

Crossmatch entre fontes SDSS do \starlight{} e do Galex. Adicionar alguns
problemas astronômicos.


%***************************************************************%
%                                                               %
%               Introdução - Organização da tese                %
%                                                               %
%***************************************************************%

\section{Organização deste trabalho}
\label{sec:Introducao:organizacao}




% End of this chapter
