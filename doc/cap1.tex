%%%%%%%%%%%%%%%%%%%%%%%%%%%%%%%%%%%%%%%%%%%%%%%%%%%%%%%%%%%%%%%%%
% Dissertacao de Mestrado / Dept Fisica, CFM, UFSC              %
% Andre@UFSC - 2011                                             %
%%%%%%%%%%%%%%%%%%%%%%%%%%%%%%%%%%%%%%%%%%%%%%%%%%%%%%%%%%%%%%%%%

%:::::::::::::::::::::::::::::::::::::::::::::::::::::::::::::::%
%                                                               %
%                          Cap�tulo 1                           %
%                                                               %
%:::::::::::::::::::::::::::::::::::::::::::::::::::::::::::::::%

%***************************************************************%
%                                                               %
%                         Introdu��o                            %
%                                                               %
%***************************************************************%

\chapter{Introdu��o}
\label{sec:Intro}

Bancos de dados em astronomia. 

%***************************************************************%
%                                                               %
%                 Introdu��o - STARLIGHT + SDSS                 %
%                                                               %
%***************************************************************%

\section{\starlight + SDSS}
\label{sec:Intro:Starlight}

Falar do sucesso do \starlight. Necessidade de ir para outros $\lambda$, e qual
ci�ncia pode ser feita com cada faixa.


%***************************************************************%
%                                                               %
%                     Introdu��o - GALEX                        %
%                                                               %
%***************************************************************%

\section{GALEX}
\label{sec:Intro:Galex}

Resumo do Galex, o que �, como funciona, motiva��o.


%***************************************************************%
%                                                               %
%           Introdu��o - STARLIGHT: Trab. anteriores            %
%                                                               %
%***************************************************************%

\section{Trabalhos Anteriores}
\label{sec:Intro:TrabAnt}

Observat�rios virtuais. Crossmatch. Galex papers.


%***************************************************************%
%                                                               %
%                 Introdu��o - Este trabalho                    %
%                                                               %
%***************************************************************%

\section{Este Trabalho}
\label{sec:Intro:EsteTrab}

Crossmatch entre fontes SDSS do \starlight e do Galex. Adicionar alguns
problemas astron�micos.


%***************************************************************%
%                                                               %
%               Introdu��o - Organiza��o da tese                %
%                                                               %
%***************************************************************%

\section{Organiza��o deste trabalho}
\label{sec:Introducao:organizacao}



% End of this chapter
