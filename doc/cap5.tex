%%%%%%%%%%%%%%%%%%%%%%%%%%%%%%%%%%%%%%%%%%%%%%%%%%%%%%%%%%%%%%%%%
% Dissertacao de Mestrado / Dept Fisica, CFM, UFSC              %
% Andre@UFSC - 2011                                             %
%%%%%%%%%%%%%%%%%%%%%%%%%%%%%%%%%%%%%%%%%%%%%%%%%%%%%%%%%%%%%%%%%

%:::::::::::::::::::::::::::::::::::::::::::::::::::::::::::::::%
%                                                               %
%                          Capítulo 5                           %
%                                                               %
%:::::::::::::::::::::::::::::::::::::::::::::::::::::::::::::::%

%***************************************************************%
%                                                               %
%                           Conclusao                           %
%                                                               %
%***************************************************************%

\chapter{Conclusões e perspectivas}
\label{sec:conclusao}

%***************************************************************%
%                                                               %
%                    Conclusao - este trabalho                  %
%                                                               %
%***************************************************************%

%***************************************************************%
%                                                               %
%              Análise - separação entre RG e PG                %
%                                                               %
%***************************************************************%

\section{A separação entre galáxias aposentadas e passivas}

Sequência de classes no diagrama cor--cor.

% FIXME: Separacão RG e RG
FIXME:\\
Tecnicamente, as galáxias passivas (PG) e as aposentadas (RG) deveriam ser
essencialmente as mesmas, com as RG apresentando linhas de emissão fracas. Em
geral as propriedades físicas destas duas classes se sobrepõem, e são distintas
das duas classes AGN. Ainda assim, é preciso levar em conta que existem
diferenças sistemáticas entre as PG e as RG.


%TODO: Comparar galáxias aposentadas com passivas
TODO: Comparar galáxias aposentadas com passivas no diagrama cor--cor.

Nas figuras \ref{fig:ATFluxColor} até \ref{fig:AVColor} pode-se notar que a
distribuição de RG tem um espalhamento considerável na direção das AGN\ldots


\textbackslash citep{[In prep. perpetuum]}\{Mateus2013\}


\section{Este trabalho}

Este estudo teve como objetivo explorar o uso de bancos de dados em astronomia.
{\em Mega-surveys} estão gerando um dilúvio de dados, é de extrema importância
que sejam desenvolvidas de técnicas e ferramentas adequadas para absorver
este volume de dados.

O uso de bancos de dados relacionais facilitou muito o acesso e o gerenciamento
dos dados de um {\em survey}. As diversas ferramentas disponíveis permitem que
se manipule os dados de uma forma eficiente.

Importação dos dados do \starlight para um banco de dados relacional.

Crossmatch com o \galex.

Como um exemplo do uso

Estudo das propriedades físicas de uma amostra de galáxias com relação às suas
cores UV.


Esta análise serviu para ilustrar a facilidade de se obter uma amostra
partindo de um banco de dados obedecendo certos critérios.


%***************************************************************%
%                                                               %
%                  Conclusao - trabalhos futuros                %
%                                                               %
%***************************************************************%

\section{Trabalhos futuros}

% TODO: Trabalhos futuros.
Expandir banco de dados do \starlight para o IR (WISE).

Estudo mais profundo da amostra \starlightUV. Examinar a contaminação entre as
classes RG e PG.

Ainda poderia haver mais trabalho no servidor de banco de dados. Muita coisa foi
executada no meu computador. A classificação e alguma análise básica podem ser
feitas de forma simples pendurando scripts no casjobs.

Mais o que?

% End of this chapter
