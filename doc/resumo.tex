%%%%%%%%%%%%%%%%%%%%%%%%%%%%%%%%%%%%%%%%%%%%%%%%%%%%%%%%%%%%%%%%%
% Dissertacao de Mestrado / Dept Fisica, CFM, UFSC              %
% Andre@UFSC - 2011                                             %
%%%%%%%%%%%%%%%%%%%%%%%%%%%%%%%%%%%%%%%%%%%%%%%%%%%%%%%%%%%%%%%%%

% Cid corrigiu e devolveu em Mar/06/2006 


%***************************************************************%
%                                                               %
%                          Resumo                               %
%                                                               %
%***************************************************************%

\begin{abstract}[Resumo]

% TODO: Resumo. 
Com o advento dos mega-surveys, e o imenso volume de dados gerados por eles,
surge a necessidade de desenvolver técnicas de data mining. Um exemplo bastante
conhecido é programa STARLIGHT, que tem como entrada uma base de populações
estelares simples e espectros de galáxias do Sloan Digital Sky Survey (SDSS), e
gera como saída informações relacionadas ao histórico de formação estelar das
galáxias. Baseada inteiramente no espectro óptico, esta análise serve de guia
para interpretar dados em outras faixas espectrais para galáxias de diferentes
tipos (como as que estão formando estrelas, as galáxias ativas de diferentes
classes, as passivas e as aposentadas). O atual data release do Galaxy Evolution
Explorer (GALEX) inclui o crossmatch com os objetos do SDSS. Isso permite
correlacionar as informações resultantes do STARLIGHT com dados de fotometria no
ultravioleta. Este trabalho\ldots

\end{abstract}

% End of resumo
