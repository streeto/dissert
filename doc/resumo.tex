%%%%%%%%%%%%%%%%%%%%%%%%%%%%%%%%%%%%%%%%%%%%%%%%%%%%%%%%%%%%%%%%%
% Dissertacao de Mestrado / Dept Fisica, CFM, UFSC              %
% Andre@UFSC - 2011                                             %
%%%%%%%%%%%%%%%%%%%%%%%%%%%%%%%%%%%%%%%%%%%%%%%%%%%%%%%%%%%%%%%%%

%***************************************************************%
%                                                               %
%                          Resumo                               %
%                                                               %
%***************************************************************%

\begin{abstract}[Resumo]

Megalevantamentos de dados astronômicos estão gerando um volume imenso de dados,
e com isso surge a necessidade de desenvolver formas de armazenamento e técnicas
de mineração de dados. Neste trabalho explora-se o uso de bancos de dados
relacionais (RDBMS) para facilitar o acesso e o gerenciamento dos dados gerados
por um levantamento de dados de grande porte. Em particular, descreve-se a
importação para um banco de dados do catálogo de propriedades físicas de
galáxias obtido através do código \starlight. O \starlight é um código de
síntese espectral, que tem como entrada uma base de populações estelares simples
e espectros de quase $1\,000\,000$ de galáxias do {\em Sloan Digital Sky Survey}
(SDSS), e gera como saída informações relacionadas ao histórico de formação
estelar das galáxias. Baseada inteiramente no espectro óptico, esta análise
serve de guia para interpretar dados em outras faixas espectrais para galáxias
de diferentes tipos (como as que estão formando estrelas, as galáxias ativas de
diferentes classes, as passivas e as aposentadas). O atual {\em data release} do
{\em Galaxy Evolution Explorer} (\galex GR6) inclui a identificação cruzada de
seus objetos com os objetos do SDSS. Isso permite correlacionar as informações
resultantes do \starlight com dados de fotometria no ultravioleta. Uma amostra
de galáxias limitada em {\em redshift} é analisada com base no diagrama de
classificação $\nII/\Halpha$ (WHAN), avaliando as suas propriedades físicas em
relação às cores em ultravioleta.

\end{abstract}

% End of resumo
